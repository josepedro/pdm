\chapter{Animations of the 1-SSD motion}
\label{appendix:animations}
The objective of this appendix is to clarify the 1-SSD motions. Therefore, an animation of a 1-SSD motions is presented in Figure \ref{figura:animate_motions}. Also, an animation of feasible mechanism 19 is presented in Figure \ref{figura:animate_mech_19}. Finally, mechanism 19 is shown adjusting the synchrony between needle and looper motions in Figure \ref{figura:animate_mech_19_adj}.

However, to visualise the animations comprised in this dissertation, it is necessary to view the \textit{Portable Document Format} file using \textit{Adobe Reader}. The animations were successfully tested on \textit{Adobe Reader}'s version 8.1.7 using \textit{Ubuntu 12.04} and \textit{Windows 7}. The digital version of this dissertation can be downloaded from the university library's website.

Access the website in \textit{http://150.162.1.90/pergamum/biblioteca/index.php} and search for the author's name.

\begin{figure}[h]
\centering
\animategraphics[autoplay, loop, width=7cm, height=7cm]{4}{figuras/anima-mov/}{1}{19}
\caption{Motions to stitch using a 1-SSD.}
\label{figura:animate_motions}
\end{figure}

\begin{figure}[h]
\centering
\animategraphics[autoplay, loop, width=10cm]{8}{figuras/anima-mec/}{1}{34}
\caption{Mechanism 19 executing a stitch.}
\label{figura:animate_mech_19}
\end{figure}

\begin{figure}[h]
\centering
\animategraphics[autoplay, loop, width=10cm]{8}{figuras/anima-ajuste/}{1}{14}
\caption{Mechanism 19 adjusting needle and looper synchrony.}
\label{figura:animate_mech_19_adj}
\end{figure}